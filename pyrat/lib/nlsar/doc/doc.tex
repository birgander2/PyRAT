%%
%% doc.tex: Documentation of NL-SAR Toolbox
%%
%% This file is part of NL-SAR Toolbox version 0.6.
%%
%% Copyright Charles-Alban Deledalle (2013)
%% Email charles-alban.deledalle@math.u-bordeaux1.fr
%%
%% This software is a computer program whose purpose is to provide a
%% suite of tools to manipulate SAR images.
%%
%% This software is governed by the CeCILL license under French law and
%% abiding by the rules of distribution of free software. You can use,
%% modify and/ or redistribute the software under the terms of the CeCILL
%% license as circulated by CEA, CNRS and INRIA at the following URL
%% "http://www.cecill.info".
%%
%% As a counterpart to the access to the source code and rights to copy,
%% modify and redistribute granted by the license, users are provided only
%% with a limited warranty and the software's author, the holder of the
%% economic rights, and the successive licensors have only limited
%% liability.
%%
%% In this respect, the user's attention is drawn to the risks associated
%% with loading, using, modifying and/or developing or reproducing the
%% software by the user in light of its specific status of free software,
%% that may mean that it is complicated to manipulate, and that also
%% therefore means that it is reserved for developers and experienced
%% professionals having in-depth computer knowledge. Users are therefore
%% encouraged to load and test the software's suitability as regards their
%% requirements in conditions enabling the security of their systems and/or
%% data to be ensured and, more generally, to use and operate it in the
%% same conditions as regards security.
%%
%% The fact that you are presently reading this means that you have had
%% knowledge of the CeCILL license and that you accept its terms.
%%
%%
%% Started on  Wed Jul 24 16:23:36 2013 Charles-Alban Deledalle
%% Last update Thu Apr 10 17:58:28 2014 Charles-Alban Deledalle
%%

\documentclass[10pt,french,english,a4paper]{article}

\usepackage[utf8]{inputenc}
\usepackage{graphicx}
\graphicspath{{images/}}
\usepackage{fancyvrb}
\usepackage{amsmath}
\usepackage{url}
\usepackage{color}
\usepackage[bookmarks,bookmarksnumbered,bookmarksopen,
  colorlinks,citecolor=blue]{hyperref}

\usepackage[vmargin=2cm,hmargin=2cm]{geometry}

\newcommand{\up}[1]{$^{#1}$}
\newcommand{\down}[1]{$_{#1}$}

\setlength{\parindent}{0cm}

\title{%
  NL-SAR Toolbox Manual
}
\author{
  Charles-Alban Deledalle\\
  \url{charles-alban.deledalle@math.u-bordeaux1.fr}
}
\date{\today}

\begin{document}
\sloppy

\maketitle

NL-SAR Toolbox provides a collection of tools for the estimation of
multi-modal SAR images with non-local filters. Beyond estimation,
NL-SAR Toolbox provides a suite of tools to manipulate SAR images.
There are 6 ways to interact with NL-SAR:
\begin{itemize}
\item through 5 interfaces:
\begin{itemize}
\item in command line
\item with Matlab
\item with Python
\item with IDL
\item as a dynamic library ({\it e.g.} to use it from a C program).
\end{itemize}
\item and through 1 plugin:
\begin{itemize}
\item for PolSARpro (experimental).
\end{itemize}
\end{itemize}

Another plugin for OTB should appear soon.\\

So far, the command line version is the most stable one while others can crash,
for instance, if you do not provide the good inputs in arguments. Feel free to fix
such bugs or contribute to NL-SAR Toolbox as you wish under the term of the license
(see Section \ref{sec:license}).\\

NL-SAR Toolbox has been compiled and tested successfully on
GNU/Linux (Ubuntu 12.04 Precise Pangolin \& Linux Mint 13 Maya) and Max OS X (10.7.5 Lion).

\tableofcontents

\section{License}
\label{sec:license}

This software is a computer program whose purpose is to provide a
suite of tools to manipulate SAR images.\\

This software is governed by the CeCILL license under French law and
abiding by the rules of distribution of free software. You can use,
modify and/ or redistribute the software under the terms of the CeCILL
license as circulated by CEA, CNRS and INRIA at the following URL
"http://www.cecill.info".\\

As a counterpart to the access to the source code and rights to copy,
modify and redistribute granted by the license, users are provided only
with a limited warranty and the software's author, the holder of the
economic rights, and the successive licensors have only limited
liability.\\

In this respect, the user's attention is drawn to the risks associated
with loading, using, modifying and/or developing or reproducing the
software by the user in light of its specific status of free software,
that may mean that it is complicated to manipulate, and that also
therefore means that it is reserved for developers and experienced
professionals having in-depth computer knowledge. Users are therefore
encouraged to load and test the software's suitability as regards their
requirements in conditions enabling the security of their systems and/or
data to be ensured and, more generally, to use and operate it in the
same conditions as regards security.\\

The fact that you are presently reading this means that you have had
knowledge of the CeCILL license and that you accept its terms.

\section{Installation}

In the following, we are assuming that you are using \texttt{bash}.
In this case, make sure that the configuration file \texttt{.bashrc} is sourced when
opening a new shell instance. This is the default behavior, but not everytime
(for instance on Mac OS X). In this case type once:
\begin{Verbatim}[frame=single]
  > echo 'source $HOME/.bashrc' >> $HOME/.bash_profile
\end{Verbatim}
% $
If you are using another shell (like \texttt{csh} or \texttt{zsh}) replace
in the followings \texttt{.bashrc} with the corresponding configuration file
and adjust the commands accordingly.

\subsection{Dependencies}

First check the following dependencies:
\begin{center}
  \begin{tabular}{|l|l|l|}
    \hline
    software		& feature & status\\
    \hline
    gcc			& to compile the project & required\\
    fftw3f		& to enable non-local filtering with fft implementation and car filters & required\\
    gcc version $> 4.2$   & to enable parallelization with OpenMP & optional\\
    lapack		& to enable non-local filtering with covariance matrices higher than 3 x 3 & optional\\
%    blas		& to enable non-local filtering with covariance matrices higher than 3 x 3 & optional\\
%    gsl			& to enable non-wishart distribution [exerimental] & optional\\
%    gslcblas		& to enable non-wishart distribution [exerimental] & optional\\
    idl			& to enable IDL interface & optional\\
    matlab		& to enable Matlab interface & optional\\
    python		& to enable Python interface & optional\\
    numpy		& to enable Python interface & optional\\
    pdflatex		& to create the documentation & optional\\
    \hline
  \end{tabular}
\end{center}
The above pieces of software have to be present in your environment
variable \texttt{PATH} (for binaries) or \texttt{LD\_LIBRARY\_PATH} (for libraries) otherwise
their associated feature will be disabled. For instance, if you want to enable the MATLAB
interface, typing in your shell:
\begin{Verbatim}[frame=single]
  > which matlab
\end{Verbatim}
should give you the path where MATLAB is installed (e.g.,
\texttt{/Applications/MATLAB\_R2013a.app/bin/matlab}). If it is not the case,
type something like:
\begin{Verbatim}[frame=single]
  > echo 'export PATH=/Applications/MATLAB_R2013a.app/bin/:$PATH' >> $HOME/.bashrc
  > source $HOME/.bashrc
\end{Verbatim}
% $
Or, if you are using aliases, you can instead do the following:
\begin{Verbatim}[frame=single]
  > echo 'export PATH=$(dirname `echo ${BASH_ALIASES[matlab]}`):$PATH' >> $HOME/.bashrc
  > source $HOME/.bashrc
\end{Verbatim}
% $

Once you have checked your dependencies, you can compile and install NL-SAR in two ways:
as a super user or as a non super user.

\subsection{Installation for super users}

First configure and compile NL-SAR by typing in a shell prompt:
\begin{Verbatim}[frame=single]
  > ./configure
  > make
  > sudo make install
\end{Verbatim}
This will install NL-SAR's command line interface and
dynamic library interface in \texttt{/usr/},
NL-SAR's Matlab interface in Matlab's subdirectory \texttt{toolbox/nlsar/},
NL-SAR's Python interface in Python's site-packages directory
and NL-SAR's IDL interface in IDL's subdirectory \texttt{external/lib/nlsar/}.\\

\subsection{Installation for non super users}

First configure and compile NL-SAR by typing in a shell prompt:
\begin{Verbatim}[frame=single]
  > ./configure --prefix=<NLSAR_PATH> --prefix-matlab=<MATLAB_PATH>
                --prefix-python=<PYTHON_PATH> --prefix-idl=<IDL_PATH>
  > make
  > make install
\end{Verbatim}
This will install NL-SAR's command line interface and
dynamic library interface in \texttt{<NLSAR\_PATH>},
NL-SAR's Matlab interface in \texttt{<MATLAB\_PATH>/toolbox/nlsar/},
NL-SAR's Python module in \texttt{<PYTHON\_PATH>}
and NL-SAR's IDL interface in \texttt{<IDL\_PATH>/lib/nlsar/}.\\

You will need to update your environment paths variables.
Make sure you are placed in the NL-SAR's directory and type the followings:
\begin{Verbatim}[frame=single]
  > echo 'export PATH=<NLSAR_PATH>/bin:$PATH' >> $HOME/.bashrc
  > echo' export LD_LIBRARY_PATH=<NLSAR_PATH>/lib:$LD_LIBRARY_PATH' >> $HOME/.bashrc
  > source $HOME/.bashrc
\end{Verbatim}
% $

\subsection{Installation of the PolSARpro plugin}

To also install the PolSARpro plugin, you just have
to adds the following option \texttt{--prefix-polsarpro=<POLSARPRO\_PATH>} to
the configure script
where \texttt{<POLSARPRO\_PATH>} is the directory where PolSARpro is installed.
For instance, if you are installing the NL-SAR Toolbox
as super-user, do the following
\begin{Verbatim}[frame=single]
  > ./configure --prefix-polsarpro=<POLSARPRO_PATH>
  > make
  > sudo make install
\end{Verbatim}
So far, the plugin is only available for PolSARpro version 4.2.0.

\subsection{Update Matlab environment}

If you are using Matlab and you have specified a different prefix during configuration,
you will need to update the Matlab environment as follows:
\begin{Verbatim}[frame=single]
  > echo 'export MATLABPATH=<MATLAB_PATH>/toolbox/nlsar/:$MATLABPATH' >> $HOME/.bashrc
  > source $HOME/.bashrc
\end{Verbatim}
% $

\subsection{Update Python environment}

If you are using Python and you have specified a different prefix during configuration,
you will need to update the Python environment as follows:
\begin{Verbatim}[frame=single]
  > echo 'export PYTHONPATH=<PYTHON_PATH>:$PYTHONPATH' >> $HOME/.bashrc
  > source $HOME/.bashrc
\end{Verbatim}
% $

\subsection{Update IDL environment}

If you are using IDL and you have specified a different prefix during configuration,
you will need to update the IDL environment as follows:
%  > echo export LD_LIBRARY_PATH=<IDL_PATH>/lib/nlsar/:'$LD_LIBRARY_PATH' >> $HOME/.bashrc
\begin{Verbatim}[frame=single]
  > echo "PREF_SET, 'IDL_PATH', '<PREFIX_IDL>/lib/nlsar/:<IDL_DEFAULT>', /COMMIT" | idl
  > echo "PREF_SET, 'IDL_DLM_PATH', '<IDL_PATH>/dlm/:<IDL_DLM_DEFAULT>', /COMMIT" | idl
  > source $HOME/.bashrc
\end{Verbatim}
% $

\section{Prerequisites}

\begin{itemize}
\item From command line: works out of the box
\item From Matlab: works out of the box
\item From Python: add the following before doing anything
\begin{Verbatim}[frame=single]
    import nlsartoolbox as nlsartb
\end{Verbatim}
\item From IDL: works out of the box
\item From C: add the following before everything
  \begin{Verbatim}[frame=single]
    #include <nlsartoolbox.h>
  \end{Verbatim}
  Link to NL-SAR Toolbox with the option \texttt{-lnlsartoolbox}
\end{itemize}

The PolSARpro plugin should work out of the box.

\section{Images formats}

Supported formats:
\begin{itemize}
  \item RAT formats,
  \item PolSARpro formats,
  \item XIMA formats.
\end{itemize}

Note that
NL-SAR deals only with images of intensity or of covariance matrices.
Other inputs will not produce what you want.
If you have amplitude or complex images, use
the program \texttt{sarjoin} which build an intensity image or
an image of covariance matrices from amplitude or complex images
(see Section \ref{sec:join}).\\

Files can be encoded in little or big endian.
The defaults behavior of NL-SAR is to interpret binary data according to your own architecture.
You can specify NL-SAR that the data uses the other endianness by adding '[SWAP]' at the
end of the file name (see Section \ref{sec:convert})

Obviously, the PolSARpro plugin deals only with PolSARpro formats.

\subsection{RAT formats}

\begin{itemize}
\item
  NL-SAR can read RAT files of version 1 and 2, but it writes only RAT files in version 1.
\item
  A RAT file is assumed to be an image of complex covariance matrices.
  A RAT file containing vectorial data will produce an error message.
  Only the arrays of the following types are implemented so far:
  \begin{itemize}
    \item float (\texttt{var} = 4)
    \item float complex (\texttt{var} = 6)
    \item double complex (\texttt{var} = 9)
  \end{itemize}
  Rat files with other types will produce an error message.
  Fell free to contact me to extend to other modalities.
\end{itemize}

\subsection{PolSARpro format}


\begin{itemize}
\item
  NL-SAR can read Sinclar, Coherency and Complex matrices,
  but it writes only Coherency and Complex matrices.
  Fell free to contact me to extend to other modalities.
\item
  A PolSARPRo data is a directory containing binary files (with extensions \texttt{.bin})
  and a \texttt{config.txt} file,
  You can find more details about this format there:
  \url{http://earth.eo.esa.int/polsarpro/Manuals/PolSARpro_DataFormat.pdf}
\item
  Note that NL-SAR can deal with a specific mode (\texttt{PolarType:~pp0})
  for reading 1-dimensional data.
\end{itemize}

\subsection{XIMA format}

\begin{itemize}
\item
  An XIMA data is a binary file without header which comes with a file with same name and extension .dim.
  You can find more details about this format there: \url{http://perso.telecom-paristech.fr/~nicolas/XIMA/index.html}.
\item
  NL-SAR can read all XIMA formats (NB: all formats haven't been tested,
  contact me if you have any troubles)
  but write only in cxf (complex float) from a
  1-dimensional SAR image.
\end{itemize}

\section{Interfaces and their basic commands}

This section describes the different basic commands/functions that are available
to manipulate SAR images in different languages: command line,
Matlab, Python, IDL or in C.

\subsection{Reading information}

\begin{itemize}
\item From command line:
  \begin{Verbatim}[frame=single]
    > sarinfo file.rat
    dimensions:
	M = 512
	N = 256
	D = 3
  \end{Verbatim}
\item From Matlab:
  \begin{Verbatim}[frame=single]
    > [M, N, D] = sarinfo('file.rat')
    M =
         512
    N =
         256
    D =
           3
  \end{Verbatim}
\item From Python:
  \begin{Verbatim}[frame=single]
    In [1]: M, N, D = nlsartb.sarinfo('file.rat')

    In [2]: M, N, D
    Out[2]: (512, 256, 3)
  \end{Verbatim}
\item From IDL:
  \begin{Verbatim}[frame=single]
    > PRINT, sarinfo('file.rat')
              512         256          3
  \end{Verbatim}
\item From C:
  \begin{Verbatim}[frame=single]
    sardata* sarimage;

    // anything

    if (!(sarimage = sardata_alloc()))
        // treat error
    if (!(sarimage = sarread_header("file.rat", sarimage)))
        // treat error
    printf("M=%d N=%d D=%d\n", sarimage->M, sarimage->N, sarimage->D);

    // anything

    sardata_free(sarimage);
  \end{Verbatim}
\end{itemize}

\subsection{Reading data}

The following commands import a SAR image from disk to memory
\begin{itemize}
\item From Matlab:
  \begin{Verbatim}[frame=single]
    > sarimage = sarread('file.rat');
  \end{Verbatim}
  Look at the matrix dimensions:
  \begin{Verbatim}[frame=single]
    > size(sarimage)
    ans =
         3      3    256    512
  \end{Verbatim}
\item From Python:
  \begin{Verbatim}[frame=single]
    In [1]: sarimage = nlsartb.sarread('file.rat')

  \end{Verbatim}
  Look at the matrix dimensions:
  \begin{Verbatim}[frame=single]
    In [2]: shape(sarimage)
    Out[2]: (512, 256, 3, 3)
  \end{Verbatim}
\item From IDL:
  \begin{Verbatim}[frame=single]
    > sarimage = sarread('file.rat')
  \end{Verbatim}
  Look at the matrix dimensions:
  \begin{Verbatim}[frame=single]
    > PRINT, size(sarimage, /DIMENSIONS)
           3           3         256         512
  \end{Verbatim}
\item From C:
  \begin{Verbatim}[frame=single]
    sardata* sarimage;

    // anything

    if (!(sarimage = sardata_alloc()))
        // treat error
    if (!(sarimage = sarread("file.rat", sarimage)))
        // treat error

    // anything

    sardata_free(sarimage);
  \end{Verbatim}
\end{itemize}

Note that a command line version would be meaningless.

\subsection{Writing data}

The following commands export a SAR image from memory to disk.
\begin{itemize}
\item From Matlab:
  \begin{Verbatim}[frame=single]
    > sarwrite(sarimage, 'newfile.rat');
  \end{Verbatim}
\item From Python:
  \begin{Verbatim}[frame=single]
    In [1]: nlsartb.sarwrite(sarimage, 'newfile.rat')
    Out[1]: True
  \end{Verbatim}
\item From IDL:
  \begin{Verbatim}[frame=single]
    > sarwrite, sarimage, 'newfile.rat'
  \end{Verbatim}
\item From C:
  \begin{Verbatim}[frame=single]
    sardata*    sarimage;

    // anything

    if (!(sarwrite(sarimage, "newfile.rat")))
        // treat error

    // anything

    sardata_free(sarimage);
  \end{Verbatim}
\end{itemize}

Note that a command line version would be meaningless.

\subsection{Join}
\label{sec:join}

The following commands creates an intensity image
or a covariance matrix from a list of amplitude images or
a single look complex images:
\begin{itemize}
\item From command line:
  \begin{Verbatim}[frame=single]
    > sarjoin file1 [file2 ... fileN] newfile
  \end{Verbatim}
\end{itemize}

Note that it is the only command of NL-SAR which deals
with amplitude or single look complex data as input. If you provide
only one file in input, this function basically compute
the intensity image from the amplitude or complex image.

\subsection{Conversion}
\label{sec:convert}

The following example convert a RAT file to PolSARpro format:
\begin{itemize}
\item From command line
  \begin{Verbatim}[frame=single]
    > sarconvert file.rat newfile
  \end{Verbatim}
\item From Matlab:
  \begin{Verbatim}[frame=single]
    > sarimage = sarread('file.rat');
    > sarwrite(sarimage, 'newfile');
  \end{Verbatim}
\item From Python:
  \begin{Verbatim}[frame=single]
    In [1]: sarimage = nlsartb.sarread('file.rat')

    In [2]: nlsartb.sarwrite(sarimage, 'newfile')
    Out[2]: True
  \end{Verbatim}
\item From IDL:
  \begin{Verbatim}[frame=single]
    > sarimage = sarread('file.rat')
    > sarwrite, sarimage, 'newfile'
  \end{Verbatim}
\item From C:
  \begin{Verbatim}[frame=single]
    sardata*    sarimage;

    // anything

    if (!(sarimage = sardata_alloc()))
        // treat error
    if (!(sarimage = sarread("file.rat", sarimage)))
        // treat error
    if (!(sarwrite(sarimage, "newfile")))
        // treat error

    // anything

    sardata_free(sarimage);
  \end{Verbatim}
\end{itemize}

The following example convert a PolSARpro format to the same PolSARpro format
except it switches the endianness of the output:
\begin{itemize}
\item From command line
  \begin{Verbatim}[frame=single]
    > sarconvert file newfile[SWAP]
  \end{Verbatim}
\item Same principle from other interfaces.
\end{itemize}

\subsection{Extraction}

The following commands extract a subarea
from position $(x, y)$ to position $(x + width - 1, y + height - 1)$
with a decimation step:
\begin{itemize}
\item From command line
  \begin{Verbatim}[frame=single]
    > sarextract file.rat newfile.rat x y width height step
  \end{Verbatim}
\item From Matlab:
  \begin{Verbatim}[frame=single]
    > sarimage_new = sarimage(:, :, y + (1:step:height)), x + (1:step:width));
  \end{Verbatim}
\item From Python:
  \begin{Verbatim}[frame=single]
    In [1]: sarimage_new = sarimage[x:x + width:step, y:y + height:step, :, :]

  \end{Verbatim}
\item From IDL:
  \begin{Verbatim}[frame=single]
    > sarimage_new = sarimage[*, *, y:(y + height - 1):step, x:(x + width - 1):step]
  \end{Verbatim}
\item From C:
  \begin{Verbatim}[frame=single]
    sardata*    sarimage;
    sardata*    sarimage_new;
    long int    xoffset, yoffset, width, height, step;

    // anything

    if (!(sarimage_new = sardata_alloc()))
        // treat error
    if (!(sarimage_new = sardata_extract(sarimage, sarimage_new,
                                         x, y, width, height, step)))
        // treat error

    // anything

    sardata_free(sarimage_new);
  \end{Verbatim}
\end{itemize}

\subsection{RGB export}

\begin{itemize}
\item From command line:
  \begin{Verbatim}[frame=single]
    > sar2png file.rat rgbexport.png [alpha]
  \end{Verbatim}
  where \texttt{alpha} is an optional parameter
  to enhance contrast (default 3)
\item From Matlab:
  \begin{Verbatim}[frame=single]
    > rgbexport = sar2rgb(sarimage [, alpha]);
  \end{Verbatim}
  The argument \texttt{alpha} is the same as for the command line version.\\
  Look at the matrix dimensions:
  \begin{Verbatim}[frame=single]
    > size(sarimage)
    ans =
         3      3    256    512
    > size(rgbexport)
    ans =
       512    256      3
  \end{Verbatim}
  The storing convention for the RGB image is reversed compared
  to our usual convention to ensure compatibility with
  the Matlab Image Toolbox.
\item From Python:
  \begin{Verbatim}[frame=single]
    In [1]: rgbexport = nlsartb.sar2rgb(sarimage [, alpha]);
  \end{Verbatim}
  The argument \texttt{alpha} is the same as for the command line version.\\
  Look at the matrix dimensions:
  \begin{Verbatim}[frame=single]
    In [2]: shape(sarimage)
    Out[2]: (512, 256, 3, 3)
    In [2]: shape(rgbexport)
    Out[2]: (512, 256, 3)
  \end{Verbatim}
\item From IDL:
  \begin{Verbatim}[frame=single]
    > rgbexport = sar2rgb(sarimage [, alpha])
  \end{Verbatim}
  The argument \texttt{alpha} is the same as for the command line version.\\
  Look at the matrix dimensions:
  \begin{Verbatim}[frame=single]
    > PRINT, SIZE(sarimage, /DIMENSIONS)
           3           3         256         512
    > PRINT, SIZE(rgbexport, /DIMENSIONS)
           3         256         512
  \end{Verbatim}
\item From C:
  \begin{Verbatim}[frame=single]
    sardata*    sarimage;
    rgbdata*    rgbexport;

    // anything

    if (!(rgbexport = rgbdata_alloc()))
        // treat error
    if (!(rgbexport = sar2rgb(sarimage, rgbexport, alpha, gamma)))
        // treat error

    // anything

    rgbdata_free(rgbexport);
  \end{Verbatim}
\end{itemize}

\subsection{Viewer}

\begin{itemize}
\item From command line:
  \begin{Verbatim}[frame=single]
    > sarshow file.rat [alpha]
  \end{Verbatim}
  The argument \texttt{alpha} is the same as for the RGB export.\\
  The first time, you will probably have the following message:
  \begin{Verbatim}[frame=single]
    Please set your environment variable NLSAR_VIEWER
  \end{Verbatim}
  You need to define an environment variable \texttt{NLSAR\_VIEWER} pointing
  to your favorite image viewer. For instance for Linux, if you like
  Eye Of Gnome, type the following:
  \begin{Verbatim}[frame=single]
    > echo 'export NLSAR_VIEWER="eog -n"' >> $HOME/.bashrc
    > source $HOME/.bashrc
  \end{Verbatim}
  Or, if you prefer Konqueror
  \begin{Verbatim}[frame=single]
    > echo 'export NLSAR_VIEWER=konqueror' >> $HOME/.bashrc
    > source $HOME/.bashrc
  \end{Verbatim}
  For Mac OS X, you can use
  \begin{Verbatim}[frame=single]
    > echo 'export NLSAR_VIEWER="open -W"' >> $HOME/.bashrc
    > source $HOME/.bashrc
  \end{Verbatim}
\item From Matlab:
  \begin{Verbatim}[frame=single]
    > sarshow(sarimage [, alpha]);
  \end{Verbatim}
  The argument \texttt{alpha} is the same as for the RGB export.
\item From Python:
  \begin{Verbatim}[frame=single]
    In [1]: nlsartb.sarshow(sarimage [, alpha]);
    Out[1]: True
  \end{Verbatim}
  The argument \texttt{alpha} is the same as for the RGB export.
\item From IDL:
  \begin{Verbatim}[frame=single]
    > sarshow, sarimage [, alpha]
  \end{Verbatim}
  The argument \texttt{alpha} is the same as for the RGB export.
\end{itemize}

%\section{Image generation}

%\subsection{Noise generation}

%\section{Image statistics}

%\subsection{Histogram of amplitudes}

%\subsection{Log-cumulant diagram}

%\subsection{Log-cumulant RGB output}

\section{Interfaces for image filtering}

\subsection{Car filters}

NL-SAR Toolbox implements the boxcar, diskcar and gausscar filters. Examples:

\begin{itemize}
\item From command line:
  \begin{Verbatim}[frame=single]
    > sarboxcar file.rat newfile.rat [hW]
  \end{Verbatim}
  where \texttt{hW} is the half-width of the box (default 1).
\item From Matlab:
  \begin{Verbatim}[frame=single]
    > sarimage_new = sarboxcar(sarimage [, hW]);
  \end{Verbatim}
  The arguments are the same as for the command line version.
\item From Python:
  \begin{Verbatim}[frame=single]
    In [1]: sarimage_new = nlsartb.sarboxcar(sarimage [, hW]);

  \end{Verbatim}
  The arguments are the same as for the command line version.
\item From IDL:
  \begin{Verbatim}[frame=single]
    > sarimage_new = sarboxcar(sarimage [, hW])
  \end{Verbatim}
  The arguments are the same as for the command line version.
\item From C:
  \begin{Verbatim}[frame=single]
    sardata*    sarimage;
    sardata*    sarimage_new;
    int         hW;

    // anything

    if (!(sarimage_new = sardata_alloc()))
        // treat error
    if (!(sarimage_new = sarboxcar(sarimage, sarimage_new, hW)))
        // treat error

    // anything

    sardata_free(sarimage_new);
  \end{Verbatim}
\end{itemize}

\subsection{NL-SAR filter}

NL-SAR Toolbox implements the NL-SAR filter. Examples:

\begin{itemize}
\item From command line:
  \begin{Verbatim}[frame=single]
    > sarnlsar file.rat newfile.rat L [hW hP verbose noise.rat]
  \end{Verbatim}
  where \texttt{L} is the equivalent number of look of the input noisy image,
  \texttt{hW} is the radius of the largest search window size (default 12) and
  \texttt{hP} the half-width of the largest patches (default 5).
  If \texttt{verbose} = 1, steps and progressing bars are displayed on the standard output (default 1).
  \texttt{noise.rat} an homogeneous image containing only noise (default iid L looks Wishart)
\item From Matlab:
  \begin{Verbatim}[frame=single]
    > sarimage_new = sarnlsar(sarimage, L [, verbose, hW, hP, sarnoise]);
  \end{Verbatim}
  The arguments are the same as for the command line version.
\item From Python:
  \begin{Verbatim}[frame=single]
    In [1]: sarimage_new = nlsartb.sarnlsar(sarimage, L [, verbose, hW, hP, sarnoise])

  \end{Verbatim}
  The arguments are the same as for the command line version.
\item From IDL:
  \begin{Verbatim}[frame=single]
    > sarimage_new = sarnlsar(sarimage, L [, verbose, hW, hP, sarnoise])
  \end{Verbatim}
  % $
  The arguments are the same as for the command line version.
\item From C:
  \begin{Verbatim}[frame=single]
    sardata*    sarimage;
    sardata*    sarimage_new;
    int         L;
    int         nb_optional_args = 2;
    int         verbose, hW;

    // anything

    if (!(sarimage_new = sardata_alloc()))
        // treat error
    if (!(sarimage_new = sarnlsar(sarimage, sarimage_new, L,
                                  nb_optional_args, verbose, hW)))
        // treat error

    // anything

    sardata_free(sarimage_new);
  \end{Verbatim}
  The arguments are the same as for the command line version.
  In this examples only two optionals arguments are given.
\end{itemize}

\paragraph*{Example}

\begin{itemize}
\item From command line:
  \begin{Verbatim}[frame=single]
    # Create a simulated polsar image with 3-looks Wishart speckle
    > sarmire example.rat 256 256 3 3
    > sarshow example.rat
  \end{Verbatim}
  \includegraphics[width=0.3\linewidth]{nlsar_example.png}
  \begin{Verbatim}[frame=single]
    # Run the non-local estimation
    > sarnlsar example.rat result.rat 3
    Load image
    Noise analysis
    	eta2=0.33 [theoretical=0.33] corr=0.03
    	=> #scales=3 step=1
    Compute similarity statistics
    	s=1 hP= 1 mean=9.758588 std=1.576362
    	s=1 hP= 5 mean=9.717225 std=0.428879
    	s=2 hP= 1 mean=1.118128 std=0.216376
    	s=2 hP= 5 mean=1.111482 std=0.064944
    	s=3 hP= 1 mean=0.375286 std=0.107277
    	s=3 hP= 5 mean=0.374484 std=0.035698
    Computation (#proc=1, #windows=1)
    |==================================================| 100%
    Save image
    > sarshow result.rat
  \end{Verbatim}
  \includegraphics[width=0.3\linewidth]{nlsar_result.png}
\item From other interfaces:
  See provided examples in directory \texttt{examples}.
\end{itemize}


\section{Plugin of the NL-SAR filter for PolSARpro}

From PolSARpro, you can run the NL-SAR filter from the menus:\\

\quad \texttt{Process} $\to$ \texttt{Polarimetric Speckle Filter}
$\to$ \texttt{NLSAR Speckle Filter}.\\

Note that unlike the interfaces described in the previous function,
the PolSARpro plugin does not give you (yet) the possibility to provide
an homogeneous image containing only noise
to the NL-SAR filter. In this case, the NL-SAR
filter will assume that the speckle is described by a L looks Wishart
distribution. Performances can then be affected, essentially when speckle
is spatially correlated.\\

This plugin is experimental and as not been tested exhaustively.

\section{Frequently asked questions (FAQ)}

\paragraph{\bf How can I participate to the development of NL-SAR Toolbox?}

Send me an email!

\paragraph{\bf Why does Matlab crash with the following message ``Invalid MEX-file [...]: undefined symbol: fftwf\_cleanup''?}

Matlab might use its own precompiled version of FFTW3. NL-SAR Toolbox uses the version
usually placed in the directory \texttt{/usr/lib/}. In order to use the right version,
run matlab from command line as follows:

  \begin{Verbatim}[frame=single]
    LD_PRELOAD=/usr/bin/libfftw3f.so:$LD_PRELOAD matlab
  \end{Verbatim}


\paragraph{\bf Why \texttt{mex} compilation failed with message \texttt{ld: library not found -lgomp}?}

Matlab might not use \texttt{gcc} to compile \texttt{C} programs.
To change this behavior, do the following

  \begin{Verbatim}[frame=single]
    cp <PREFIX_MATLAB>/bin/mexopts.sh $HOME/.matlab/<version>/mexopts.sh
    chmod 755 $HOME/.matlab/<version>/mexopts.sh
  \end{Verbatim}

where you replace \texttt{<PREFIX\_MATLAB>} by the directory where MATLAB is installed
and \texttt{<version>} by the version identifier (something like \texttt{R2013a}).
Then edit the file \texttt{\$HOME/.matlab/<version>/mexopts.sh}. Locate the lines

  \begin{Verbatim}[frame=single]
    CC='<some compiler>'
  \end{Verbatim}

where \texttt{<some compiler>} might be for instance ``\texttt{xcrun -sdk macosx10.7 clang}''
and replace them by

  \begin{Verbatim}[frame=single]
    CC='gcc'
  \end{Verbatim}

Then rerun compilation.

\paragraph{\bf I've just installed some dependencies (for instance \texttt{libfftw3f})
  but when I run \texttt{./configure} it cannot find it, why?}

NL-SAR Toolbox uses the command \texttt{locate} to find your dependencies.
You may need to update your locate database. On Linux run

  \begin{Verbatim}[frame=single]
    sudo updatedb
  \end{Verbatim}

On Mac OS X run

  \begin{Verbatim}[frame=single]
    sudo /usr/libexec/locate.updatedb
  \end{Verbatim}

\paragraph{\bf
Most of your examples are based on covariance matrices.
Could you give an example of using the NL-SAR filter with an image of intensity?}

Intensity images are particular cases of $1 \times 1$ matrices. Hence, every examples
with covariance matrices can be used identically with intensity images.
File with RAT or XIMA formats can be used for intensity images.
A modified PolSARpro format can be used for intensity image with
as specific mode \texttt{PolarType: pp0}.
If you are using another specific format, an alternative is to load your file
from Matlab, Python, IDL or C in a single precision floating matrix that you give as an argument
to NL-SAR functions.
Recall that NL-SAR Toolbox assumes that the content of a scalar image corresponds to
intensity values (not the amplitude). You can use
\texttt{sarjoin} (see Section \ref{sec:join}) to convert amplitude images
to intensity images.

\paragraph{\bf I've exported an intensity image in PolSARpro format
  from the Next ESA SAR Toolbox (NEST), but NL-SAR Toolbox cannot read it, why?}

When you export an intensity image from NEST to PolSARpro format, the directory contains
a file \texttt{config.txt} and \texttt{Intensity\_HH.bin}.
Rename \texttt{Intensity\_HH.bin} to \texttt{C11.bin} and change
in the file \texttt{config.txt} the field \texttt{PolarType} from
\texttt{pp1} to \texttt{pp0}.
NL-SAR Toolbox will then read your file successfully, otherwise contact me.

\paragraph{\bf I want to produce interferograms with NL-SAR toolbox
but the input and output are images of covariance matrices. What should I do?}

From a pair of single look complex images (SLC), you can build an image of $2 \times 2$
covariance matrix with the command \texttt{sarjoin} (see Section \ref{sec:join}).
Then run NL-SAR on this image. After you can retrieve the interferometric phase or
coherence (normalized correlation) from the produced image. For instance,
you can do the following using Matlab:

  \begin{Verbatim}[frame=single]
    > insar_phase     = squeeze(angle(sarimage(1,2,:,:)));
    > insar_coherence = squeeze(abs(sarimage(1,2,:,:)) ./ ...
                          sqrt(abs(sarimage(1,1,:,:)) .* abs(sarimage(2,2,:,:)));
  \end{Verbatim}

If you moreover assume the same reflectivity (radar cross-section) for each pair
of corresponding pixels in both images,
the coherence can be refined as

  \begin{Verbatim}[frame=single]
    > insar_coherence = squeeze(2 * abs(sarimage(1,2,:,:)) ./ ...
                          (abs(sarimage(1,1,:,:)) + abs(sarimage(2,2,:,:)));
  \end{Verbatim}

Look at the matrix dimensions:
  \begin{Verbatim}[frame=single]
    > size(sarimage)
    ans =
         2      2    256    512
    > size(insar_phase)
    ans =
       256    512
    > size(insar_coherence)
    ans =
       256    512
  \end{Verbatim}

\section{Not documented yet}

\begin{itemize}
\item
\texttt{sarnlsar\_dispatch} for distributing NL-SAR on a cluster.
\item
\texttt{sarcat} for concatenating files.
\item
\texttt{sarmire} to generate a simulated SAR image with multi-scale structure.
\item
\texttt{sarnoise} to generate an homogeneous SAR image with noise only.
\end{itemize}

\end{document}
